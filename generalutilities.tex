\section{Utilities}
    \subsection{Binomialkoeffizienten}
        $\binom{n}{k} = \frac{n!}{k!\cdot(n-k)!}$
    \subsection{Bekannte Ableitungen}
        $\ln(x)' = \frac{1}{x} $ \\
        $\sin'{x} = \cos{x}$ \\
        $\cos'{x} = -\sin{x}$ \\
        $\tan'{x}=\frac{1}{\cos{x}^{2}} = 1 + \tan{x}^{2}$ \\
        $\arcsin'{x} = \frac{1}{\sqrt{1-x^{2}}}$ \\
        $\arccos'{x} = -\arcsin'{x}$ \\
        $\arctan'{x} = \frac{1}{1+x^{2}}$ \\
        $\sinh'{x} = \cosh{x}$ \\
        $\cosh'{x} = \sinh{x}$ \\
        $\tanh'{x} = \frac{1}{\cosh{x}^{2}} = 1 - \tanh{x}^{2}$ \\
        $\left(\frac{x^{\frac{3}{2}}}{\frac{3}{2}}\right)' = \sqrt{x}$
    \subsection{DGL Ansätze}
        \subsubsection{Homogen}
            $y(t) = e^{\lambda t}$
        \subsubsection{Inhomogen}
            $K(t)$ sei der Störfaktor (0 bei homogener DGL).
            \begin{tabular}{|p{2cm}|p{2.5cm}|p{3.6cm}|}
                \hline
                K(t) & Bedingung & Ansatz für $y_p(t)$ \\
                \hline
                \multirow{2}{*}{$t^r$} & 0 keine 0-Stelle & $A_0+a_1t+\dotsb+A_rt^r$ \\ \cline{2-3}
                & 0 $m$-fache 0-Stelle & $A_0t^m+A_11t^{m+1}+\dotsb+A_rt^{m+r}$ \\
                \hline
                $b_0+b_1t+\dotsb+b_rt^r, b_i\in\mathbb{R}$ & 0 keine 0-Stelle & $A_0+A_1t+\dotsb+A_rt^r$ \\
                \hline
                \multirow{2}{*}{$e^{\lambda_0 t}, \lambda_0 \in \mathbb{C}$} & $\lambda_0$ keine 0-Stelle & $Ae^{\lambda_0 t}$ mit $A=\frac{1}{\lambda^n+a_{n-1}\lambda^{n-1}+\dotsb+a_1\lambda+a_0}$ \\ \cline{2-3}
                & $\lambda_0$ $m$-fache 0-Stelle & $At^me^{\lambda_0t}$ \\
                \hline
                \multirow{2}{*}{$\cos(\omega t), \sin(\omega t)$} & $\pm i\omega$ keine 0-Stelle & $A\cos(\omega t) + B\sin(\omega t)$ \\ \cline{2-3}
                & $\pm i\omega$ mehrfache 0-Stelle & $t(A\cos(\omega t)+B\sin(\omega t))$ \\
                \hline
                $t^2e^{-t}$ & $-1$ keine 0-Stelle & $(A_0+A_1t + A_2t^2)e^{-t}$ \\
                \hline
            \end{tabular}
    \subsection{Begriffe}
        Bijektiv: umkehrbar eindeutig (surjektiv \& injektiv) $\forall y \in B:\exists! x \in A: f(x) = y$\\
        Surjektiv: jeder Wert der Zielbereich (Wertemenge) wird angenommen $\rightarrow$ Wertemenge = Bildmenge $\forall y \in B: \exists x \in A: f(x) = y$\\
        Injektiv: jeder Wert der Bildmenge (Lösungsmenge) wird genau einmal angenommen $\forall x, x' \in A: f(x) = f(x') \Rightarrow x = x'$\\
        Disjunkt: Komplett verschieden \\
        Monotonie: $\forall x, x' \in I; x < x' \Rightarrow f(x) \leqslant f(x')$ \\
        Streng monoton: $\forall x, x' \in I: x < x' \Rightarrow f(x) < f(x')\Rightarrow$ bijektiv;  $f'(x) > 0\Rightarrow$ streng monoton wachsend
    \subsection{Trigonometrie \/ Goniometiere}
        $\sinh^2x-\cosh^2x=1$ \\
        $\sin^2x+\cos^2x=1$ \\
        $\sin{x}\cos{y} = \frac{1}{2}(\sin(x - y)+\sin(x + y))$ \\
        $\sin{x}\sin{x} = \frac{1-\cos(2x)}{2}$ \\
        $\cos(\arcsin(x)) = \sqrt{1-x^2}$ \\
        $\sin(2 \arcsin(x)) = 2x\sqrt{1-x^2}$ \\
        $\sin(x\pm y) = \sin{x}\cos{y}\pm\sin{y}\cos{x}$ \\
        $\cos(x\pm y) = \cos{x}\cos{y}\mp\sin{x}\sin{y}$ \\
        $\sin(2x) = 2\sin{x}\cos{x} = \frac{2\tan{x}}{1+\tan^2(x)}$ \\
        $\cos(2x) = \cos^2(x)-\sin^2(x) = 1-2\sin^2(x)$ \\
        $\sin^2(x) = \frac{1}{2}(1-\cos(2x))$ \\
        $\cos^2(x) = \frac{1}{2}(1+\cos(2x))$ \\
        $\sin{x}+\sin{y} = 2\sin\left(\frac{x+y}{2}\right)\cos\left(\frac{x-y}{2}\right)$ \\
        $\sin{x}-\sin{y} = 2\cos\left(\frac{x+y}{2}\right)\sin\left(\frac{x-y}{2}\right)$ \\
        $\cos{x}+\cos{y} = 2\cos\left(\frac{x+y}{2}\right)\cos\left(\frac{x-y}{2}\right)$ \\
        $\cos{x}-\cos{y} = 2\sin\left(\frac{x+y}{2}\right)\sin\left(\frac{x-y}{2}\right)$
	\subsubsection{Funktionswerte}
		\begin{tabular}{l|l l l l l}
			$\varphi$ & 0° & 30° & 45° & 60° & 90° \\
			\hline
			$\sin{\varphi}$ & $0$ & $\frac{1}{2}$ & $\frac{\sqrt{2}}{2}$ & $\frac{\sqrt{3}}{2}$ & $1$ \\
			$\cos{\varphi}$ & $1$ & $\frac{\sqrt{3}}{2}$ & $\frac{\sqrt{2}}{2}$ & $\frac{1}{2}$ & $0$ \\
			$\tan{\varphi}$ & $0$ & $\frac{\sqrt{3}}{3}$ & $1$ & $\sqrt{3}$ & - \\
		\end{tabular}
    \subsection{Geometrie}
        \subsubsection{Kreis}
            Kreismenge: $\{x^2+y^2\leq R^2\}$ \\
            Fläche: $A=\pi r^2$\\
            Umfang: $U=2\pi r$
        \subsubsection{Kugel}
            Fläche: $A=4\pi r^2$ \\
            Volumen: $\frac{4}{3}\pi r^3$
        \subsubsection{Dreiecke}
            $r = \frac{abc}{4a} = \frac{a}{2\sin{\alpha}}$: Umkreisradius \\
            $s = \frac{a+b+c}{2}$ \\
            $A = \frac{1}{2}ah = \frac{abc}{4r} = \sqrt{s(s-a)(s-b)(s-c)}$
            \begin{theorem}[Cosinussatz]
                $a^{2} = b^{2} + c^{2} + 2bc\cos{\alpha}$
            \end{theorem}
            \begin{theorem}[Sinussatz]
                $\frac{a}{\sin{\alpha}} = 2r$
            \end{theorem}
            \paragraph{Gleichschenkliges Dreieck}
                $h = r = \frac{\sqrt{2}}{2}a, A = \frac{1}{2}a^{2}$
            \subsubsubsection{Gleichseitiges Dreieck}
                $h = \frac{\sqrt{3}}{2}a, a= \frac{\sqrt{3}}{4}a^{2}$
            \subsubsubsection{Rechtwinkliges Dreieck}
                $h^{2} = pq, a^{2} = pc, b^{2} = qc, h = \frac{ab}{c}$
        \subsubsection{Vierecke}
            \subsubsubsection{Quadrat}
                $d = \sqrt{2}a, r = \frac{d}{2}$
            \subsubsubsection{Rechteck}
                Diagonalen $e = f= \sqrt{a^{2} + b^{2}}$
            \subsubsubsection{Parallelogramm}
                $e^{2} + f^{2} = 2(a^{2} + b^{2})$ \\
                $A= ab\sin{\alpha} = ah_{a} = bh_{b}$
    \subsection{Physikalische Grössen}
        Masse mit Dichte $\mu$: $m(x) = \int_{X} \mu (x) \, \mathrm{dvol}_n(x)$ \\
        Schwerpunkt: $S(x) = m(x)^{-1} \int_{X} x\mu(x) \, \mathrm{dvol}_n (x)$ \\
        Drehmoment: $M= J\alpha$ \\
        Trägheitsmoment: $J = \int_{X} \vec{r}^2_{\bot} \mu(x) \, \mathrm{d} x$
    \subsection{Einheiten}
        $\num{e5} \si{\pascal} = \num{1} \si{\bar}$ \\
        $\num{1} \si{atm} = \num{1.01325} \si{\bar} = \num{101325} \si{\pascal}$ \\
        $\num{1} \si{\bar} = \num{14.5} \si{PSI}$ \\
        $\num{1} \si{\kilo\watt\hour} = \num{3.6e6} \si{\joule}$ \\
        $\num{0} \si{\degreeCelsius} = \num{273.2} \si{\kelvin}$
        \subsubsection{Grössenbeschreiber}
            \begin{tabular}{l l | l l}
                \si{\atto} & \num{e-18} & \si{\exa} & \num{e18} \\
                \si{\femto} & \num{e-15} & \si{\peta} & \num{e15} \\
                \si{\pico} & \num{e-12} & \si{\tera} & \num{e12} \\
                \si{\nano} & \num{e-9} & \si{\giga} & \num{e9} \\
                \si{\micro} & \num{e-6} & \si{\mega} & \num{e6} \\
                \si{\milli} & \num{e-3} & \si{\kilo} & \num{e3} \\
                \si{\centi} & \num{e-2} & \si{\hecto} & \num{e2} \\
                \si{\deci} & \num{e-1} & \si{\deca} & \num{e1} \\
            \end{tabular}
    \subsection{Vektoren}
        Skalarprodukt: $\vec{a}\cdot\vec{b}=x_a\cdot x_b+y_a\cdot y_b$ etc. \\
        Vektorprodukt: $\vec{a}\times\vec{b}= \begin{pmatrix}
            a_2b_3 - a_3b_2 \\
            a_3b_1 - a_1b_3 \\
            a_1b_2 - a_2b_1
        \end{pmatrix}$ \\
        $|\vec{a} \times \vec{b}| = |\vec{a}||\vec{b}|\sin{\varphi}$ \\
        $\mathrm{det}\begin{pmatrix}
            a & b \\
            c & d
        \end{pmatrix} = ad - bc$
    \subsection{Koordinaten}
        \subsubsection{Kartesische}
            $x = r\cos\varphi = r\cos\rho\cos\varphi$ \\
            $y = r\sin\varphi = r\sin\rho\sin\varphi$ \\
            $z = r\sin\rho$
        \subsubsection{Zylinder}
            $r=\sqrt{x^2+y^2}; \varphi = \arctan{\frac{x}{y}}; z$
        \subsubsection{Kugel}
            $r=\sqrt{x^2+y^2+z^2}; \varphi = \arctan{\frac{x}{y}}; \rho$
    \subsection{Bekannte Reihen}
        $\displaystyle (1+x)^a = \sum_{k=0}^{\infty}\binom{a}{k}x^k$ \\
        $\displaystyle \frac{1}{1-\frac{z}{c}} = \sum_{k=0}^{\infty}\left(\frac{z}{c}\right)^k$ für $|z| < c$ \\
        $\displaystyle \sin{x} = \sum_{k=0}^{\infty}\frac{(-1)^k\cdot x^{2k+1}}{(2k+1)!}$ \\
        $\displaystyle \cos{x} = \sum_{k=0}^{\infty}\frac{(-1)^k\cdot x^{2k}}{(2k)!}$ \\
        $\displaystyle \sinh{x} = \sum_{k=0}^{\infty}\frac{x^{2k+1}}{(2k+1)!}$ \\
        $\displaystyle \cosh{x} = \sum_{k=0}^{\infty}\frac{x^{2k}}{(2k)!}$ \\
        $\displaystyle e^z = \sum_{k=0}^{\infty}\frac{z^k}{k!}$ \\
        $\displaystyle \ln(1+x) = \sum_{k=1}^{\infty}\frac{(-1)^{k+1}x^k}{k!}$
    \subsection{Komplexe Zahlen}
        $(a+\imath b)(c+\imath d)=(ac-bd)+\imath(ad+bc)$ \\
        $\displaystyle\frac{a+\imath b}{c+\imath d}=\frac{ac+bd}{c^2+d^2}+\imath\frac{bc-ad}{c^2+d^2}$ \\
        $\displaystyle \sqrt[n]{z}=\sqrt[n]{|z|}\cdot e^{\imath\left(\frac{\varphi}{n}+\frac{2\pi k}{n}\right)}$ \\
        $z^n=r^ne^{\imath n\varphi}$ \\
        $\frac{1}{\imath} = -\imath$
    \subsection{Logarithmen}
        $\log_b(x^n) = n\log_b(x)$ \\
        $\log_b(x\cdot y) = \log_b(x) + \log_b(y)$ \\
        $\log_b(\frac{x}{y}) = \log_b(x) - \log_b(y)$ \\
        $\log_b(x) = \frac{\log_n(x)}{\log_n(b)}$
